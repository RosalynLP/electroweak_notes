\documentclass[a4paper,12pt]{article}
\usepackage{fullpage} % Package to use full page
\usepackage[utf8]{inputenc}
%\setlength\parindent{0pt}
\usepackage{amsmath} %for writing text in equations with \text{}
\usepackage{dsfont} %for identity matrix
\usepackage[T1]{fontenc}
\usepackage{slashed}
\usepackage{tikz}

\newcommand{\icol}[1]{% inline column vector
  \left(\begin{smallmatrix}#1\end{smallmatrix}\right)%
}

\newcommand{\irow}[1]{% inline row vector
  \begin{smallmatrix}(#1)\end{smallmatrix}%
}

\title{\textbf{Gauge Theories: Electroweak}}
\author{Lecturer: Richard Ball}
\date{\normalsize\today} % Can't work out how to make this a UK format but oh well

\begin{document}

\maketitle
\section{Introduction}
%
Imagine a world without electroweak:
\begin{itemize}
    \item Still have electromagnetism (EM), massive hadrons, atoms, gravity etc.
    \item Parity (P) and charge conjugation (C) are still good symmetries
    \item Flavour is always conserved: everything lasts forever (and always existed)
    \item No neutrinos (would be non-interacting)
\end{itemize}
%
Neutrinos were first hypothesised by Pauli (1930) to explain the missing energy in $\beta$-decay. They were first observed in 1956 (Cowan-Reines). Parity violation was first directly observed in 1956/7 (Lee-Yang-Wu).

\subsection{Chirality}
%
Define the projection operators $P_R \equiv \frac{1}{2}(1 + \gamma^5)$ and $P_L \equiv \frac{1}{2}(1 - \gamma^5)$. Recalling that $(\gamma^5)^2 = 1$ and $(\gamma^5)^\dagger = \gamma^5$, we can deduce the following properties:
\begin{itemize}
    \item $P_R^2 = P_R$
    \item $P_L^2 = P_L$
    \item $P_LP_R = P_RP_L = 0$
    \item $P_R + P_L = 1$
    \item $P_R - P_L = \gamma^5$.
\end{itemize}
% 
Any Dirac spinor can be split up into a right-handed and a left-handed component, $\psi = \psi_R + \psi_L$, using the projection operators to define $\psi_R = P_R \psi$ and $\psi_L = P_L \psi$. Since $\gamma^\mu P_L = P_R \gamma^\mu$, it follows that $\bar{\psi} P_R \equiv (\bar{\psi})_R = \bar{\psi}_L$ and $\bar{\psi} P_L \equiv (\bar{\psi})_L = \bar{\psi}_R$. So 
\begin{equation}
\begin{split}
    \bar{\psi}\psi = \bar{\psi}(\psi_R + \psi_L) &= \bar{\psi}_L\psi_R + \bar{\psi}_R\psi_L \\
    &\text{and} \\
    \bar{\psi}\gamma^\mu \psi = \bar{\psi}\gamma^\mu(\psi_R + \psi_L) &= \bar{\psi}_R\gamma^\mu\psi_R 
    \bar{\psi}_L\gamma^\mu\psi_L.
\end{split}
\end{equation}
%
The Dirac Lagrangian splits up like 
\begin{equation}
\begin{split}
\mathcal{L}_D &= \bar{\psi}(i\slashed{\partial}-m)\psi \\ 
&= \bar{\psi}_R i\slashed{\partial}\psi_R + \bar{\psi}_L i\slashed{\partial}\psi_L
+ m(\bar{\psi}_L\psi_R + \bar{\psi_R}\psi_L)
\end{split}
\end{equation}
so the mass term mixes $\psi_R$ and $\psi_L$: if $m \to 0$, $\psi_R$ and $\psi_L$ are independent. In this scenario they are both 2-component spinors obeying the Weyl equation $i\slashed{\partial}\psi_{R/L} = 0$.

\subsection{Helicity}
The spin operator can be expressed as $\Sigma^i = \frac{i}{2} \epsilon^{ijk}[\gamma^j, \gamma^k] = \gamma^5\gamma^0\gamma^i$. Then $[P_L, \Sigma^i] = [P_R, \Sigma^i]$: spin and chirality commute. Now consider \textit{helicity}, defined 
\begin{equation}
    h \equiv \frac{2\underline{\Sigma} \cdot \underline{p}}{|\underline{p}|}.
\end{equation}
$h$ has eigenvalues $\pm1$, which follows from $(\slashed{p} - m)u^\pm = 0 \implies hu^\pm = \pm u^\pm$. But $\slashed{p} = E\gamma^0 - \underline{\gamma}\cdot\underline{p}$ and $h = \gamma^5\gamma^0 (\underline{\gamma}\cdot\underline{p})/|\underline{p}| = \gamma^5\gamma^0(E\gamma^0 - \slashed{p})$. So
%
\begin{equation}
\begin{split}
\gamma^5(E - \gamma^0\slashed{p})u^\pm &= \pm u^\pm \\
\implies (P_R-P_L)(E-\gamma^0m)u^\pm &= \pm p(P_R + P_L)u^\pm \\
\text{ and } (E \mp p)u_R^\pm &= m\gamma^0 u_L^\pm \\
             (E \pm p)u_L^\pm &= m\gamma^0 u_R^\pm. 
\end{split}
\end{equation}
Again, the mass term mixes R and L, but if $m \to 0$, $p = E + \mathcal{O}(\frac{m^2}{E})$ and $2E u_R^- = 2E u_L^+ = 0$ so $u_R^- = u_L^+ = 0$, i.e. $u_R$ has helicity +1 and $u_L$ has helicity -1. 

Note that when $m=0$, helicity is Lorentz invariant (no rest frame). For $m \neq 0$, 
\begin{equation}
\begin{split}
    u_R^- = \frac{m\gamma^0}{E+p}u_L^- &\approx \frac{m}{2E}\gamma^0u_L^- \\
    \text{ and } u_L^+ &\approx \frac{m}{2E}\gamma^0u_R^+.
\end{split}
\end{equation}
You get similar expressions for negative energy solutions:
\begin{equation}
\begin{split}
\text{for $m = 0$:   } &v_R^- = v_L^+ = 0 \\
\text{for $m \neq 0$:    }  &v_R^- = -\frac{m\gamma^0}{2E}v_L^- \text{ and } v_L^+ = -\frac{m\gamma^0}{2E}v_R^-.
\end{split}
\end{equation}
%
\subsection{The Chiral Representation}
%
In the chiral representation the gamma matrices can be expressed as:
\[ \gamma^0 = \left( \begin{array}{cc}
0 & 1  \\
1 & 0  \end{array} \right) \qquad
\gamma^i = \left( \begin{array}{cc}
0 & -\sigma^i  \\
\sigma^i & 0  \end{array} \right) \qquad
\gamma^5 = \left( \begin{array}{cc}
1 & 0 \\
0 & -1     \end{array} \right)\] 
%
so
%
\[ P_R = \left( \begin{array}{cc}
1 & 0  \\
0 & 0  \end{array} \right) \qquad
P_L = \left( \begin{array}{cc}
0 & 0  \\
0 & 1  \end{array} \right) \qquad
\implies
\psi = \left( \begin{array}{c}
\psi_R  \\
\psi_L      \end{array} \right).\] 
%
Furthermore, the spin operator can be written
%
\[ \Sigma^i = \gamma^5\gamma^0\gamma^i = \left( \begin{array}{cc}
\sigma^i & 0  \\
0 & \sigma^i  \end{array} \right) \]
%
so the helicity eigenstates are $\icol{\psi_R^\pm\\0}$ and $\icol{0\\\psi_L^\pm}$. Now consider the positive and negative energy solutions. 
%
$(\slashed{p} - m)u =0 \implies$
%
\[ \left( \begin{array}{cc}
-m & E +\underline{\sigma}\cdot\underline{p}  \\
  E -\underline{\sigma}\cdot\underline{p}& -m  \end{array} \right)
\left( \begin{array}{c}
u_R   \\
u_L  \end{array} \right) = 0.\] 
%
But $\underline{\sigma}\cdot \underline{p}\ u_{L/R} = \pm p\ u_{L/R}^\pm$ so $(E \pm p)u_L^\pm = m\ u_R^\pm$. Using $E^2 = p^2 + m^2$:
\begin{equation}
\begin{split}
(E \pm p)u_L^\pm &= \sqrt{(E+p)(E-p)}u_R^\pm \\
\implies \sqrt{E\pm p}\ u_L^\pm &= \sqrt{E \mp p}\ u_R^\pm 
\end{split}
\end{equation}
and we can write
%
\[ u^\pm =
\left( \begin{array}{c}
\sqrt{E \pm p} \ \xi^\pm   \\
\sqrt{E \mp p} \ \xi^\pm  \end{array} \right) \] 
%
where $\xi^+ = \icol{1\\0}$ and $\xi^- = \icol{0\\1}$. 
\newline
\newline
\textit{Exercise: check that the normalisations $\bar{u}u = 2m$, $u^\dagger u = 2E$.}
\newline

Similarly, $(\slashed{p} +m)v = 0$ and $(\underline{\sigma}\cdot\underline{p})v^\pm = \mp p\ v^\pm$, so 
\begin{equation}
\begin{split}
(E \mp p)v_L^\pm &= -m\ v_R^\pm \\
\implies \sqrt{E\mp p}\ v_L^\pm &= -\sqrt{E \pm p}\ v_R^\pm 
\end{split}
\end{equation}
and
%
\[ v^\pm =
\left( \begin{array}{c}
\sqrt{E \mp p} \ \xi^\pm   \\
-\sqrt{E \pm p} \ \xi^\pm  \end{array} \right). \] 
%
You can relate $\xi^\pm$ to $\chi^\pm$ by charge conjugation: $v^\pm = i \gamma^2 u^{\pm *}$, where
%
\[ i\gamma^2 = \left( \begin{array}{cc}
0 & -i \sigma^2  \\
i\sigma^2 & 0  \end{array} \right); \qquad 
-i\sigma^2 = \left( \begin{array}{cc}
0 & -1 \\
1 & 0 \end{array} \right) \]
%
so $\chi^\pm = -i\sigma^2\xi^\pm$ and if $\xi^\pm = \icol{1\\0}, \icol{0\\1}$ then $\chi^\pm = \icol{0\\1}, \icol{-1\\0}$ as before.
%
\subsection{Parity}
%
Under P, $\psi \to \psi_p = \gamma^0\psi$. We know that $P_L \gamma^0 = \gamma^0 P_R$, so $\psi_L \to \gamma^0 \psi_L =  P_R\gamma^0 \psi_L = (\psi_p)_R$. In other words, $(\psi_L)_p = (\psi_p)_R$, so parity switches L and R.

Note that $[\gamma^0, \Sigma^i] = 0$ but under P $\underline{p} \to - \underline{p}$ so $h \to -h$, as expected. This means $u_R^+ to u_L^-$ etc.: parity flips helicity but not spin.
%
\subsection{Charge conjugation}
%
Under C, $\psi \to \psi_c = C\overline{\psi}^T$ where $C = i\gamma^2\gamma^0$. Then $P_LC = CP_L$ so $\psi_L \to C(\overline{\psi})_L^T$. 
This means that charge conjugation leaves the chirality unchanged. Helicity is also unchanged: C just takes particles $\leftrightarrow$ antiparticles.
%
\subsection{Time reversal}
%
Under T, $\psi \to \psi_T = B\psi$ where $B = i\gamma^1\gamma^3 = -i\gamma^5C$ and $B^\dagger = B = B^{-1}$. Again, $P_LB = BP_L$ so $\psi_L \to B\psi_L$ and time reversal leaves chirality and helicity unchanged (it reverses both spin and momentum). 
%
%%%%%%%%%%%%%%%%%%%%%%%%%%%%%%%%%%%%%%%%%%%%%%%%%%%%%%%%%%%%%%%%%%%%%%%%%%%%%%%%%%%%%%%%%%%%%%%%%%%%%%%%%
%
\section{Charged Current Electroweak}
%
\subsection{Fermi Theory (1934)}
%
Fermi theory is based on a point-like 4-fermion interaction
\begin{equation}
    G_F(\bar{n}\gamma_\mu p)(\bar{\nu}\gamma^\mu e)
\end{equation}
representing the process $n \to pe^-\bar{\nu}$, where $n$ is a neutron, $p$ is a proton, $e^-$ is an electron and $\bar{\nu}$ is an antineutrino. More generally, the interaction can be written as $G_F J_\mu^\dagger J^\mu$ where $J_\mu$ is the "weak current" and is composed of leptonic and hadronic contributions: $J_\mu = \bar{\nu}\gamma_\mu e + \bar{p}\gamma_\mu n + ...$. Note that $J_\mu$ has $\Delta Q = +1$ so $J_\mu^\dagger$ has $\Delta Q = -1$ and electric charge is conserved. 

Considering mass dimensions $[\cdot]$: $[m\psi\bar{\psi}]=+\psi$, $[\psi]=3/2$ and $[J_\mu] =3$ so $[G_F]=-2$. In other words, the coupling scales as 1/mass$^2$. The mass scale is $\approx$ 300 GeV, so the interaction is weak. The original Fermi interaction conserves parity (and C and CP), just like QED. This turned out to be wrong, according to theory developed by Lee and Yang (1956) and the Wu experiment in 1957. In weak interactions, P and C are violated by CP is conserved. More on this later.
%
\subsection{V-A Theory}
%
Developed by Masshart, Sudarshan, Feynman, Gell-Man ... in 1958. V-A theory is based on a vector current $V_\mu$ and an axial current $A_\mu$,
\begin{equation}
\begin{split}
V_\mu &= \bar{\nu}\gamma_\mu e + \bar{p}\gamma_\mu n + ... \\
A_\mu &= \bar{\nu}\gamma_\mu \gamma^5 e + \bar{p}\gamma_\mu \gamma^5 n + ...
\end{split}
\end{equation}
whose difference gives the overall current
\begin{equation}
\begin{split}
\frac{1}{2}J_\mu = \frac{1}{2}(V_\mu - A_\mu)  &= \bar{\nu}\gamma_\mu\frac{1}{2}(1-\gamma^5) e + \bar{p}\gamma_\mu\frac{1}{2}(1-\gamma^5) n + ... \\
&= \bar{\nu}_L\gamma_\mu e_L + \bar{p}_L\gamma_\mu n_L + ... .
\end{split}
\end{equation}
So the weak interactions involve \textbf{only left-handed fields}, and maximally violate P and C. In fact, under P, C and CP the currents transform in the following ways:
\begin{equation}
\begin{split}
&P \qquad V^\mu \to V_\mu, \qquad A^\mu \to -A_\mu, \qquad (V-A)^\mu \to (V+A)_\mu \\
&C \qquad V^\mu \to -V^\mu, \qquad A^\mu \to A^\mu, \qquad (V-A)^\mu \to (-V-A)^\mu \\
&CP \qquad V^\mu \to -V_\mu, \qquad A^\mu \to -A_\mu, \qquad (V-A)^\mu \to -(V-A)_\mu 
\end{split}
\end{equation}
so $(V-A)^{\mu \dagger}(V-A)_\mu$ is invariant under CP (and T). 

Note that that neutrinos \textbf{only} interact weakly, so $\nu_R$ does not interact at all! If neutrinos are massless, there is no need for $\nu_R$. From 1930-1998 neutrinos were always assumed to be massless: here we will assume $m_\nu=0$ so $\nu$ are always left handed and $\bar{\nu}$ are always right handed. In reality $m_\nu \leq$ 0.3 eV, which is very small.

So we have
%
\begin{equation}
\mathcal{L}_{4F} = -\frac{G_F}{\sqrt{2}} J_\mu^\dagger J^\mu
\end{equation}
where the $\sqrt{2}$ is historical. $J_\mu = J_\mu^l + J_\mu^h$ can be split up into a leptonic and a hadronic current, each of which comprises three generations. For example, the leptonic current
\begin{equation}
\frac{1}{2}J_\mu^l = \bar{\nu}_e\gamma_\mu e_L + \bar{nu}_{(\mu)}\gamma_\mu \mu_L + \bar{\nu}_\tau \gamma_\mu \tau_L.
\end{equation}
Lepton number $L_e = N_{e^-} - N_{e^+} - N_{\nu_e} + N_{\bar{\nu}_e}$ and the corresponding $L_\mu$ and $L_\tau$ are conserved, according to Noether's Theorem.  

The hadronic current
\begin{equation}
\frac{1}{2}J_\mu^h = \bar{u}_L\gamma_\mu d_L^\prime + \bar{c}_L \gamma_\mu s_L^\prime + \bar{t}_L \gamma_\mu b_L^\prime
\end{equation}
is simplest to express in terms of the quarks. The baryon number $B = \sum_{gen} (N_u -N_{\bar{u}} - N_d + N_{\bar{d}})$ is conserved. Things are complicated due to the effects of quark mixing - see later.

(V-A) Theory is quite complicated, but there is only one coupling, $G_F$: we call this universality. Explicitly,
\begin{equation}
\mathcal{L}_{4F} = -\frac{G_F}{\sqrt{2}} (J_\mu^{l \dagger} J^{l\mu} + (J_\mu^{h \dagger} J^{l\mu} + J_\mu^{l \dagger} J^{h\mu}) + J_\mu^{h \dagger} J^{h\mu} ) 
\end{equation}
so there are three types of (charged current) weak interaction
\begin{itemize}
    \item \textbf{leptonic:} only leptons (and neutrinos) e.g. $\mu \to e \nu_\mu \bar{\nu}_e$
    \item \textbf{semi-leptonic:} both leptons and quarks e.g. $\pi^- = d\bar{u} \to \mu^- \bar{\nu}_\mu$
    \item \textbf{hadronic:} only quarks e.g. $\Lambda \to p \pi$
\end{itemize}
But all of them violate P and C and conserve CP, only involve left-handed particles and only involve one coupling $G_F$. 
\newline
\newline
\textbf{Example:} $\pi^+ \approx u\bar{d} \to \mu^+ \nu_\mu$. 
\newline
Fig. \ref{pioncp} demonstrates the sequential application of C and P to this process. Neither of the intermediate steps can happen, as they involve either a right-handed neutrino or a left-handed antineutrino.
\begin{figure}[h!]
\caption{Visualisation of applying C and P to the process $\pi^+ \approx u\bar{d} \to \mu^+ \nu_\mu$.}
\label{pioncp}
\centering
\begin{tikzpicture}
\draw[<-] (0,1) -- (1,1);
\node[] at (0,0.5) {$\mu^-$};
\draw[fill=black] (1.5,1) circle (0.1cm);
\node[] at (1.5,0.5) {$\pi^-$};
\draw[->] (2,1) -- (3,1) ;
\node[] at (3, 0.5){$\bar{\nu}_\mu$};
%
\draw[<-] (-4,3) -- (-3,3);
\node[] at (-4,2.5) {$\mu^+$};
\draw[fill=black] (-2.5,3) circle (0.1cm);
\node[] at (-2.5,2.5) {$\pi^+$};
\draw[->] (-2,3) -- (-1,3) ;
\node[] at (-1, 2.5){$\nu_\mu$};
%
\draw[->] (5.5,3) -- (6.5,3);
\node[] at (6.5,2.5) {$\mu^-$};
\draw[fill=black] (5,3) circle (0.1cm);
\node[] at (5,2.5) {$\pi^-$};
\draw[->] (4.5,3) -- (3.5,3) ;
\node[] at (3.5, 2.5){$\bar{\nu}_\mu$};
%
\draw[<-] (0,6) -- (1,6);
\node[] at (0,5.5) {$\nu_\mu$};
\draw[fill=black] (1.5,6) circle (0.1cm);
\node[] at (1.5,5.5) {$\pi^-$};
\draw[->] (2,6) -- (3,6) ;
\node[] at (3, 5.5){$\mu^+$};
%
\draw[->] (1.5,5) --(1.5,2) ;
\draw[->] (2.5,5) --(4.5,4) ;
\draw[->] (0.5,5) --(-1.5,4) ;
\draw[<-] (2.5,1.5) --(4.5,2) ;
\draw[<-] (0.5,1.5) --(-1.5,2) ;
%
\node[] at (2,3.5) {CP};
\node[] at (4,5) {C};
\node[] at (-1,5) {P};
\node[] at (-1,1.5) {C};
\node[] at (4,1.5) {P};
%
\draw[red,->] (2.5,1.25) -- (3,1.25);
\draw[red,->] (0,6.25) -- (0.5,6.25);
\draw[red,->] (-1.5,3.25) -- (-1,3.25);
\draw[red,->] (3.5,3.25) -- (4,3.25);
\end{tikzpicture}
\end{figure}
%
\subsection{Leptonic decays}
%
\textbf{Example:} $\mu^- \to e^- \bar{\nu}_e \nu_\mu$. 
\newline
 
\end{document}
