\section{Flavour Physics}
%
We still need a theory of the quark masses (and mixing). The third generation in the standard model consists of:
\begin{itemize}
\item \textbf{$\tau$ lepton: } discovered in 1975 at SLAC, with a mass $m_\tau = $ 1.78 GeV. This was relatively unexpected;
\item \textbf{$b$ quark: } discovered in 1977 at Fermilab, with a mass $m_b$ = 5 GeV. Seen in the form of the $\Upsilon = b\bar{b}$ particle with mass 10 GeV;
\item \textbf{$\nu_\tau$ neutrino: } finally seen in 2000 at Fermilab. This was very difficult!;
\item \textbf{$t$ quark: } finally discovered in 1995 at the Tevatron, Fermilab, with a mass $m_t$ = 175 GeV.
\end{itemize}
The top quark is very heavy: $m_t > m_W, m_Z, m_h$, and decays very quickly via $t \to Wb$ before it can hadronise. Because of anomalies it does not decouple from low energy processes; its involvement is needed for consistency.

There are only three generations (3 light neutrinos).
%
\subsection{The Quark Masses}
%
As with leptons, these must arise through Yukawa interactions. We have left-handed doublets and right-handed singlets under SU(2)$_L$:
\[ Q_L^i = \left( \begin{array}{cc}
u_L   \\
d_L  \end{array} \right), \qquad
\left( \begin{array}{cc}
c_L   \\
s_L   \end{array} \right), \qquad
\left( \begin{array}{cc}
t_L   \\
b_L   \end{array} \right);\]
\[ 
u_R^i = \qquad u_R, \qquad c_R, \qquad t_R, \qquad;
\]
\[ 
d_R^i = \qquad d^\prime_R, \qquad s^\prime_R, \qquad b_R, \qquad \text{for } i=1,2,3.
\]
where the "up-type" quarks appear on the top row and "down-type" quarks on the bottom row. For leptons we introduced Yukawa coupling to a complex doublet
\[\phi = \left( \begin{array}{cc}
\phi^+ \\
\phi^0 \end{array} \right),
\]
which can give mass to the down-type quarks through $<\phi^0> = v$. What about up-type quarks? Consider the conjugate field
\[
\phi^c \equiv (i\sigma_2)\phi^+ = \left( \begin{array}{cc}
0 & 1   \\
-1 & 0  \end{array} \right) 
\left( \begin{array}{cc}
\phi^{+ *}   \\
\phi^{0 *}  \end{array} \right) = 
\left( \begin{array}{cc}
\phi^{0 *}   \\
-\phi^{+ *}  \end{array} \right).
\]
Clearly $\phi^c$ has $Y = - \frac{1}{2}$. Moreover
\begin{equation}
(i\sigma_2)\sigma_i^* = -\sigma_i(i\sigma_2),
\end{equation}
so under SU(2)$_L$, $\phi \to U \phi = \exp(\frac{i}{2}\underline{\alpha}\cdot \underline{\sigma})$, and
\begin{equation}
\phi^c \to (i\sigma_2)e^{-\frac{i}{2}\underline{\alpha}\cdot\underline{\sigma}^*} \phi^* = e^{+\frac{i}{2}\underline{\alpha}\cdot{\sigma}}(i\sigma_2)\phi^* = U \phi^c,
\end{equation}
so $\phi^c$ is also a doublet under SU(2)$_L$.

Note that this works because SU(2) is "pseudoreal". This implies also that $tr \sigma_i \{\sigma_j, \sigma_k \} = 0$, i.e. it is anomaly free. This does not generalise to SU(N), for N>2: for this we would need a second Higgs doublet. So besides $\bar{Q}_L\phi d_R$, which gives mass to down-type quarks, we also have $\bar{Q}_L\phi^c u_R$, which gives mass to up-type quarks. (What does the missing line here say?)

So the most general Yukawa term for quarks is
\begin{equation}
\mathcal{L}_{Y} = -(Y_{ij}^d \bar{Q}_L^i \phi d_R^j + h.c.) - (Y_{ij}^u \bar{Q}_L^i \phi^c u_R^j + h.c.),
\end{equation}
where $Y^u$, $Y^d$ are two arbitrary 3$\times$3 matrices. After spontaneous symmetry breaking
\begin{equation}
\phi \to \frac{v}{\sqrt{2}}\begin{pmatrix} 0 \\ 1 \end{pmatrix},
\qquad \phi^c \to \frac{v}{\sqrt{2}}\begin{pmatrix} 1 \\ 0 \end{pmatrix}.
\end{equation}
So defining mass matrices $M_{ij}^u = \frac{v}{\sqrt{2}}Y_{ij}^u$; $M_{ij}^d = \frac{v}{\sqrt{2}}Y_{ij}^d$, we have
\begin{equation}
\mathcal{L}_{Y} = -(M_{ij}^u \bar{u}_L^i u_R^j + h.c.) - (M_{ij}^d \bar{d}_L^i d_R^j + h.c.).
\end{equation}
Note that there is mixing between up-type quarks and between down-type quarks, but not between different types: this would violate U(1)$_Y$. We need to diagonalise the mass matrices, in order to find the "mass basis". Consider $M^u$: $M^{u \dagger} M^u$ is hermitian, so the real positive eigenvalues are $m_u^2$, $m_c^2$ and $m_t^2$. Similarly, $M^{d \dagger} M^d$ is hermitian, so the real positive eigenvalues are $m_d^2$, $m_s^2$ and $m_b^2$.
\subsubsection{Lemma: $\exists$ unitary matrices $L$ and $R$ such that $LMR^\dagger=D$, a diagonal matrix with masses $m_u^2$, $m_c^2$ and $m_t^2$ (or $m_d$, $m_s^2$ and $m_b^2$).}
\textbf{Proof: }
\begin{equation}
(M^\dagger M)^\dagger = M^\dagger M, 
\end{equation}
so $\exists$ a unitary matrix $R$ such that
\begin{equation}
R^\dagger(M^\dagger M)R = D^2,
\end{equation}
where $D$ is a diagonal matrix with real, positive eigenvalues. Then define $L = MRD^{-1}$, so
\begin{equation}
L^\dagger = D^{-1}R^\dagger M^\dagger, \qquad L^\dagger L = D^{-1}R^\dagger M^\dagger MR D^{-1} = 1,
\end{equation}
i.e. $L$ is unitary and $L^\dagger MR = D^{-1}R^\dagger M^\dagger MR = D$, so we have
\begin{equation}
L^{u \dagger}M^u R^u = D^u, \qquad L^{d \dagger}M^d R^d = D^d,
\end{equation}
and if we let
\begin{equation}
\begin{split}
u_R \to R^u u_R \qquad u_L \to L^u u_L \\
d_R \to R^d d_R \qquad d_L \to L^d d_L
\end{split}
\end{equation}
then the mass terms become diagonal:
\begin{equation}
- \sum_{q = u,c,t} m_q(\bar{q}_L q_R + \bar{q}_R q_L) - \sum_{q = d,s,b} m_q(\bar{q}_L q_R + \bar{q}_R q_L) = - \sum_{q=u,c,t,d,s,b} m_q \bar{q}{q}.
\end{equation}
It is easy to see that the kinetic term remains diagonal:
\begin{equation}
\mathcal{L}_{D,\ kin} = \sum_{q = d,s,b}(\bar{q}_L i \slashed{\partial} q_L + \bar{q}_R i \slashed{\partial} q_R),
\end{equation}
and $L^u$, $L^d$, $R^u$, $R^d$ are all hermitian. Likewise, the quark contribution to the neutral current
\begin{equation}
J_\mu^{NC} = \sum_{q=u,c,t}\bar{q}\gamma_\mu\bigg(\frac{-4}{3}\sin^2\theta_w +\frac{1}{2}(1-\gamma^5)\bigg)q + \sum_{q=d,s,b}\bar{q}\gamma_\mu\bigg(\frac{2}{3}\sin^2\theta_w -\frac{1}{2}(1-\gamma^5)\bigg)q,
\end{equation}
is invariant, since 
\begin{equation}
\bar{q}\gamma^\mu q = \bar{q}_L\gamma^\mu q_L + \bar{q}_L\gamma^\mu q_R.
\end{equation}
This is the GIM mechanism, which we saw earlier in the course.
%
\subsection{The CKM Matrix}
%
Charge conjugation relates up-type quarks to down-type quarks, so 
\begin{equation}
\begin{split}
\frac{1}{2} J_\mu^h = \sum_i \bar{u}_L^i \gamma_\mu d_L^i &\to \sum_{ijk} \bar{u}_L^i(L^{u \dagger})^{ij}\gamma_\mu(L^{d})^{jk}d_L^k \\
&\equiv \sum_{ij} \bar{u}_L^i \gamma_\mu V^{ij} d_L^j,
\end{split}
\end{equation}
where $V = L^{u \dagger} L^d$ is the CKM matrix, named after Cabibbo, Kobayashi and Maskawa (1973). The CKM matrix mixes the charged current interactions.

$V$ is unitary: $V^\dagger V = L^{d \dagger}L^u L^{u \dagger} L^d = 1$. There are $N$ generations, so $V$ is an $N\times N$ complex matrix and has $2N^2$ real parameters. Unitarity gives $N^2$ constraints, so there are $N^2$ remaining real parameters. However, phase changes of the form $u_L^i \to \exp(i\alpha_i)u_L^i$, $d_L^i \to \exp(i\beta_i)d_L^i$ can remove $ 2N-1$ of these parameters. So overall we have 
\begin{equation}
N^2 - (2N-1) = (N-1)^2 = \frac{N}{2}(N-1) + \frac{1}{2}(N-1)(N-2).
\end{equation}
An $N\times N$ real orthogonal matrix has $\frac{N}{2}(N-1)$ parameters, so we have $\frac{N}{2}(N-1)$ angles and $\frac{1}{2}(N-1)(N-2)$ phases. Consdering different numbers of generations:
\begin{enumerate}
\item $N=1$:

$V=1$ and there is no mixing.
\item $N=2$:

There is one angle and are no phases.
\begin{equation}
V = \begin{pmatrix}
\cos\theta_L & \sin\theta_L \\
-\sin\theta_L & \cos\theta_c
\end{pmatrix},
\end{equation}
where $\theta_c$ is the Cabbibo angle.
\item $N=3$:

There are three angles and one phase.
\begin{equation}
V = \begin{pmatrix}
V_{ud} & V_{us} & V_{ub} \\
V_{cd} & V_{cs} & V_{cb} \\
V_{td} & V_{ts} & V_{tb} 
\end{pmatrix} =
\begin{pmatrix}
1 & 0 & 0 \\
0 & c_1 & s_1 \\
0 & -s_1 & c_1 
\end{pmatrix}
\begin{pmatrix}
c_2 & 0 & s_2e^{i\delta} \\
0 & 1 & 0 \\
-s_2 e^{i\delta} & 0 & c_2
\end{pmatrix}
\begin{pmatrix}
c_3 & s_3 & 0 \\
-s_3 & c_3 & 0 \\
0 & 0 & 1
\end{pmatrix},
\end{equation}
where we have used the notation $c_i = \cos\theta_i$, $s_i = \sin\theta_i$, with $\theta_i$ being the Euler angles and $\delta$ being the phase.
\end{enumerate}
\textbf{Notes: }
\begin{itemize}
\item We can put the mixing into down-type quarks (as is convention) $d_L^{i \prime} = V^{* ij} d_L^j$ or into up-type quarks $\bar{u}_L^{j \prime} = \bar{u}_L^iV^{ij}$, and the physics is unchanged.
\end{itemize}