\section{Flavour Physics}
%
We still need a theory of the quark masses (and mixing). The third generation in the standard model consists of:
\begin{itemize}
\item \textbf{$\tau$ lepton: } discovered in 1975 at SLAC, with a mass $m_\tau = $ 1.78 GeV. This was relatively unexpected;
\item \textbf{$b$ quark: } discovered in 1977 at Fermilab, with a mass $m_b$ = 5 GeV. Seen in the form of the $\Upsilon = b\bar{b}$ particle with mass 10 GeV;
\item \textbf{$\nu_\tau$ neutrino: } finally seen in 2000 at Fermilab. This was very difficult!;
\item \textbf{$t$ quark: } finally discovered in 1995 at the Tevatron, Fermilab, with a mass $m_t$ = 175 GeV.
\end{itemize}
The top quark is very heavy: $m_t > m_W, m_Z, m_h$, and decays very quickly via $t \to Wb$ before it can hadronise. Because of anomalies it does not decouple from low energy processes; its involvement is needed for consistency.

There are only three generations (3 light neutrinos).
%
\subsection{The Quark Masses}
%
As with leptons, these must arise through Yukawa interactions. We have left-handed doublets and right-handed singlets under SU(2)$_L$:
\[ Q_L^i = \left( \begin{array}{cc}
u_L   \\
d_L  \end{array} \right), \qquad
\left( \begin{array}{cc}
c_L   \\
s_L   \end{array} \right), \qquad
\left( \begin{array}{cc}
t_L   \\
b_L   \end{array} \right);\]
\[ 
u_R^i = \qquad u_R, \qquad c_R, \qquad t_R, \qquad;
\]
\[ 
d_R^i = \qquad d^\prime_R, \qquad s^\prime_R, \qquad b_R, \qquad \text{for } i=1,2,3.
\]
where the "up-type" quarks appear on the top row and "down-type" quarks on the bottom row. For leptons we introduced Yukawa coupling to a complex doublet
\[\phi = \left( \begin{array}{cc}
\phi^+ \\
\phi^0 \end{array} \right),
\]
which can give mass to the down-type quarks through $<\phi^0> = v$. What about up-type quarks? Consider the conjugate field
\[
\phi^c \equiv (i\sigma_2)\phi^+ = \left( \begin{array}{cc}
0 & 1   \\
-1 & 0  \end{array} \right) 
\left( \begin{array}{cc}
\phi^{+ *}   \\
\phi^{0 *}  \end{array} \right) = 
\left( \begin{array}{cc}
\phi^{0 *}   \\
-\phi^{+ *}  \end{array} \right).
\]
Clearly $\phi^c$ has $Y = - \frac{1}{2}$. Moreover
\begin{equation}
(i\sigma_2)\sigma_i^* = -\sigma_i(i\sigma_2),
\end{equation}
so under SU(2)$_L$, $\phi \to U \phi = \exp(\frac{i}{2}\underline{\alpha}\cdot \underline{\sigma})$, and
\begin{equation}
\phi^c \to (i\sigma_2)e^{-\frac{i}{2}\underline{\alpha}\cdot\underline{\sigma}^*} \phi^* = e^{+\frac{i}{2}\underline{\alpha}\cdot{\sigma}}(i\sigma_2)\phi^* = U \phi^c,
\end{equation}
so $\phi^c$ is also a doublet under SU(2)$_L$.

Note that this works because SU(2) is "pseudoreal". This implies also that $tr \sigma_i \{\sigma_j, \sigma_k \} = 0$, i.e. it is anomaly free. This does not generalise to SU(N), for N>2: for this we would need a second Higgs doublet. So besides $\bar{Q}_L\phi d_R$, which gives mass to down-type quarks, we also have $\bar{Q}_L\phi^c u_R$, which gives mass to up-type quarks. (What does the missing line here say?)

So the most general Yukawa term for quarks is
\begin{equation}
\mathcal{L}_{Y} = -(Y_{ij}^d \bar{Q}_L^i \phi d_R^j + h.c.) - (Y_{ij}^u \bar{Q}_L^i \phi^c u_R^j + h.c.),
\end{equation}
where $Y^u$, $Y^d$ are two arbitrary 3$\times$3 matrices. After spontaneous symmetry breaking
\begin{equation}
\phi \to \frac{v}{\sqrt{2}}\begin{pmatrix} 0 \\ 1 \end{pmatrix},
\qquad \phi^c \to \frac{v}{\sqrt{2}}\begin{pmatrix} 1 \\ 0 \end{pmatrix}.
\end{equation}
So defining mass matrices $M_{ij}^u = \frac{v}{\sqrt{2}}Y_{ij}^u$; $M_{ij}^d = \frac{v}{\sqrt{2}}Y_{ij}^d$, we have
\begin{equation}
\mathcal{L}_{Y} = -(M_{ij}^u \bar{u}_L^i u_R^j + h.c.) - (M_{ij}^d \bar{d}_L^i d_R^j + h.c.).
\end{equation}
Note that there is mixing between up-type quarks and between down-type quarks, but not between different types: this would violate U(1)$_Y$. We need to diagonalise the mass matrices, in order to find the "mass basis". Consider $M^u$: $M^{u \dagger} M^u$ is hermitian, so the real positive eigenvalues are $m_u^2$, $m_c^2$ and $m_t^2$. Similarly, $M^{d \dagger} M^d$ is hermitian, so the real positive eigenvalues are $m_d^2$, $m_s^2$ and $m_b^2$.
\subsubsection{Lemma: $\exists$ unitary matrices $L$ and $R$ such that $LMR^\dagger=D$, a diagonal matrix with masses $m_u^2$, $m_c^2$ and $m_t^2$ (or $m_d$, $m_s^2$ and $m_b^2$).}
\textbf{Proof: }
\begin{equation}
(M^\dagger M)^\dagger = M^\dagger M, 
\end{equation}
so $\exists$ a unitary matrix $R$ such that
\begin{equation}
R^\dagger(M^\dagger M)R = D^2,
\end{equation}
where $D$ is a diagonal matrix with real, positive eigenvalues. Then define $L = MRD^{-1}$, so
\begin{equation}
L^\dagger = D^{-1}R^\dagger M^\dagger, \qquad L^\dagger L = D^{-1}R^\dagger M^\dagger MR D^{-1} = 1,
\end{equation}
i.e. $L$ is unitary and $L^\dagger MR = D^{-1}R^\dagger M^\dagger MR = D$, so we have
\begin{equation}
L^{u \dagger}M^u R^u = D^u, \qquad L^{d \dagger}M^d R^d = D^d,
\end{equation}
and if we let
\begin{equation}
\begin{split}
u_R \to R^u u_R \qquad u_L \to L^u u_L \\
d_R \to R^d d_R \qquad d_L \to L^d d_L
\end{split}
\end{equation}
then the mass terms become diagonal:
\begin{equation}
- \sum_{q = u,c,t} m_q(\bar{q}_L q_R + \bar{q}_R q_L) - \sum_{q = d,s,b} m_q(\bar{q}_L q_R + \bar{q}_R q_L) = - \sum_{q=u,c,t,d,s,b} m_q \bar{q}{q}.
\end{equation}
It is easy to see that the kinetic term remains diagonal:
\begin{equation}
\mathcal{L}_{D,\ kin} = \sum_{q = d,s,b}(\bar{q}_L i \slashed{\partial} q_L + \bar{q}_R i \slashed{\partial} q_R),
\end{equation}
and $L^u$, $L^d$, $R^u$, $R^d$ are all hermitian. Likewise, the quark contribution to the neutral current
\begin{equation}
J_\mu^{NC} = \sum_{q=u,c,t}\bar{q}\gamma_\mu\bigg(\frac{-4}{3}\sin^2\theta_w +\frac{1}{2}(1-\gamma^5)\bigg)q + \sum_{q=d,s,b}\bar{q}\gamma_\mu\bigg(\frac{2}{3}\sin^2\theta_w -\frac{1}{2}(1-\gamma^5)\bigg)q,
\end{equation}
is invariant, since 
\begin{equation}
\bar{q}\gamma^\mu q = \bar{q}_L\gamma^\mu q_L + \bar{q}_L\gamma^\mu q_R.
\end{equation}
This is the GIM mechanism, which we saw earlier in the course.
%
\subsection{The CKM Matrix}
%
Charge conjugation relates up-type quarks to down-type quarks, so 
\begin{equation}
\begin{split}
\frac{1}{2} J_\mu^h = \sum_i \bar{u}_L^i \gamma_\mu d_L^i &\to \sum_{ijk} \bar{u}_L^i(L^{u \dagger})^{ij}\gamma_\mu(L^{d})^{jk}d_L^k \\
&\equiv \sum_{ij} \bar{u}_L^i \gamma_\mu V^{ij} d_L^j,
\end{split}
\end{equation}
where $V = L^{u \dagger} L^d$ is the CKM matrix, named after Cabibbo, Kobayashi and Maskawa (1973). The CKM matrix mixes the charged current interactions.

$V$ is unitary: $V^\dagger V = L^{d \dagger}L^u L^{u \dagger} L^d = 1$. There are $N$ generations, so $V$ is an $N\times N$ complex matrix and has $2N^2$ real parameters. Unitarity gives $N^2$ constraints, so there are $N^2$ remaining real parameters. However, phase changes of the form $u_L^i \to \exp(i\alpha_i)u_L^i$, $d_L^i \to \exp(i\beta_i)d_L^i$ can remove $ 2N-1$ of these parameters. So overall we have 
\begin{equation}
N^2 - (2N-1) = (N-1)^2 = \frac{N}{2}(N-1) + \frac{1}{2}(N-1)(N-2).
\end{equation}
An $N\times N$ real orthogonal matrix has $\frac{N}{2}(N-1)$ parameters, so we have $\frac{N}{2}(N-1)$ angles and $\frac{1}{2}(N-1)(N-2)$ phases. Consdering different numbers of generations:
\begin{enumerate}
\item $N=1$:

$V=1$ and there is no mixing.
\item $N=2$:

There is one angle and are no phases.
\begin{equation}
V = \begin{pmatrix}
\cos\theta_L & \sin\theta_L \\
-\sin\theta_L & \cos\theta_c
\end{pmatrix},
\end{equation}
where $\theta_c$ is the Cabbibo angle.
\item $N=3$:

There are three angles and one phase.
\begin{equation}
V = \begin{pmatrix}
V_{ud} & V_{us} & V_{ub} \\
V_{cd} & V_{cs} & V_{cb} \\
V_{td} & V_{ts} & V_{tb} 
\end{pmatrix} =
\begin{pmatrix}
1 & 0 & 0 \\
0 & c_1 & s_1 \\
0 & -s_1 & c_1 
\end{pmatrix}
\begin{pmatrix}
c_2 & 0 & s_2e^{i\delta} \\
0 & 1 & 0 \\
-s_2 e^{i\delta} & 0 & c_2
\end{pmatrix}
\begin{pmatrix}
c_3 & s_3 & 0 \\
-s_3 & c_3 & 0 \\
0 & 0 & 1
\end{pmatrix},
\end{equation}
where we have used the notation $c_i = \cos\theta_i$, $s_i = \sin\theta_i$, with $\theta_i$ being the Euler angles and $\delta$ being the phase.
\end{enumerate}
\textbf{Notes: }
\begin{itemize}
\item We can put the mixing into down-type quarks (as is convention) $d_L^{i \prime} = V^{* ij} d_L^j$ or into up-type quarks $\bar{u}_L^{j \prime} = \bar{u}_L^iV^{ij}$, and the physics is unchanged.
\item $V$ depends on only $L^u$ and $L^d$ (not $R^u$ and $R^d$), because charged current is left-handed.
\item For leptons, because up-type leptons (i.e. neutrinos) are massless, we can absorb $V$ into $\nu_L$ with impunity; there is no mixing. We see that mixing and mass are intimately related.
\item Mixing follows directly from spontaneous symmetry breaking for quark masses, but it's not pretty... we have 10 new parameters (6 masses, 3 angles and 1 phase).
\item The charged current Feynman rules pick up factors of elements of $V$ and $V^\dagger$, e.g.
\newline
\includegraphics[width=0.4\linewidth]{figs/46a.png}
\includegraphics[width=0.4\linewidth]{figs/46b.png}
\end{itemize}
%
\subsection{Unitarity Triangles}
%
Testing for unitarity is equialent to testing for the condition $VV^\dagger =1$. There are different ways of doing this:
\begin{enumerate}
\item \textbf{Universality tests: }
There are three tests of the form $(V^\dagger V)_{ii}=1$, e.g.
$|V_{ud}|^2 +|V_{us}|^2 +|V_{ub}|^2 = 1$.
\item \textbf{Triangle tests: }
There are six tests of the form $(V^\dagger V)_{ij}=0$ for $i \neq j$, e.g.
$V^*_{ub}V_{ud} + V^*_{cb}V_{cd} + V^*_{tb}V_{td} = 0$. 

This is a triangle because it is a sum of three complex numbers. The diagram on the right shows the different connections between the quark types.
\newline
\includegraphics[width=0.6\linewidth]{figs/47a.png}
\includegraphics[width=0.3\linewidth]{figs/47b.png}
\end{enumerate}
\textbf{Determining $|V_{ij}|$: }
\begin{itemize}
\item The top left part of $V$ (which doesn't involve $t$ or $b$ quarks) is determined by measuring $\theta_c$ as above. More specifically: $|V_{ud}|$ is determined through $\beta$ and pion decay, $|V_{us}|$ through kaon and tau decays, and $|V_{cd}|$ and $|V_{cs}|$ are determined through $D$ semileptonic decays such as $D \to \pi l \nu$ and $D \to K l \nu$.
\item Additional elements involving the $b$ quark but not the $t$ quark (i.e. "B physics") are determined by $B$ semileptonic decays such as $B \to \pi l \nu$ and $B \to D l \nu$. 
\item $|V_{tb}|$ is determined through single top producgion $t \to bW$, but $|V_{ts}|$ and $|V_{td}|$ are very hard to measure directly. We can use corrections to loop calculations to find these (see later).
\end{itemize}
The results of the combined fits are usefully summarised by the Wolfenstein parametrisation (1983):
\begin{equation}
V = \begin{pmatrix}
1-\frac{\lambda^2}{2} & \lambda & A\lambda^3(\rho - i\eta) \\
-\lambda &  1-\frac{\lambda^2}{2} & A \lambda^2\\
A\lambda^3(1-\rho-i\eta) & -A\lambda^2 & 1 
\end{pmatrix} + \mathcal{O}(\lambda^4),
\end{equation}
where $\lambda \approx \sin\theta_c \approx\ 0.225$, $A \approx\ 0.8$, $\rho \approx\ 0.12$ and $\eta \approx 0.36$ are $\mathcal{O}(1)$. Since $\lambda \gg \lambda^2 \gg \lambda^3$, this expresses a curious hierarchy of the off-diagonal elements, which is not understood! 

Considering the various universality tests and triangles, we can see that in this parametrisation $V$ is unitary.
%
\subsection{CP Violation}
%
Up until now, every term in the electroweak Lagrangian has been CP invariant. In particular, under CP $W_\mu^+ \to - W^{\mu -}$, so that the charged current interaction for leptons transforms as 
\begin{equation}
\begin{split}
\frac{g}{\sqrt{2}}(W_\mu^+\bar{\psi}_LT^-\gamma^\mu \psi_L + W_\mu^-\bar{\psi}_L T^+ \gamma^\mu \psi_L) &\to \frac{g}{\sqrt{2}}(-W^{\mu -}(-\bar{\psi}_L(T^-)^T \gamma_\mu \psi_L) - W^{\mu +}\bar{\psi}_L(T^+)^T \gamma_\mu \psi_L) \\
&= \frac{g}{\sqrt{2}}(W_\mu^- (\bar{\psi}_L T^+ \gamma^\mu \psi_L) + W_\mu^+ \bar{\psi}_L T^- \gamma^\mu \psi_L),
\end{split}
\end{equation}
since $(T^+)^T = T^-$. However, we now need to include the effects of the CKM matrix in the charged-current interaction for quarks:
\begin{equation}
\begin{split}
&\frac{g}{\sqrt{2}}(W_\mu^+\bar{d}_L\gamma^\mu V^\dagger u_L + W_\mu^-\bar{u}_L \gamma^\mu  V d_L)\\
= &\frac{g}{\sqrt{2}}(W_\mu^+\bar{Q}_L T^-\gamma^\mu V^\dagger Q_L + W_\mu^- \bar{Q}_L T^+ \gamma^\mu V Q_L)\\
\to &\frac{g}{\sqrt{2}}(W_\mu^-\bar{Q}_L (T^-)^T\gamma^\mu (V^\dagger)^T Q_L + W_\mu^+ \bar{Q}_L (T^+)^T \gamma^\mu V^T Q_L)\\
= &\frac{g}{\sqrt{2}}(W_\mu^+\bar{Q}_L T^-\gamma^\mu V^{*\dagger} Q_L + W_\mu^- \bar{Q}_L T^+ \gamma^\mu V^* Q_L).
\end{split}
\end{equation}
So we have CP invariance only if $V = V^*$. This means that the electroweak sector is CP invariant for two generations ($N=2$), but not for three generations ($N=3$) if $\delta \neq 0$. In other words, this complex phase, $\delta$, is CP-violating. Note that from the CPT theorem, which states invariance of CPT, that CP violation implies T violation. T is equivalent to complex conjugation, i.e. $\exp(iHt) \to \exp(-iHt)$.

A useful measure of the amount of CP violation is given by the area of the unitarity triangle. For example, if we consider the $bd$ triangle,
\newline
\begin{wrapfigure}{l}{0.4\linewidth}
  \centering
  \includegraphics[width=\linewidth]{figs/48a.png}
\end{wrapfigure}
where we have defined $\lambda_i = V_{ib}^* V_{id} = \lambda_i^1 + i \lambda_i^2$. The total area of the triangle is given by
\begin{equation}
\begin{split}
A &= \frac{1}{2}|\underline{\lambda}_u \wedge \underline{\lambda}_c| \\
&= \frac{1}{2}|\lambda_u^1 \lambda_c^2 - \lambda_u^2 \lambda_c^1| \\
&=\frac{1}{2}|\Im(\lambda_u^* \lambda_c)| \\
&= \frac{1}{2}|\Im(V_{ub}V^*_{ud} V_{cb} V^*_{cd})|.
\end{split}
\end{equation}
In fact, all the triangles have the same area (\textit{Exercise:} prove this). We can thus define the \textit{Jarlskog invariant} (Jarlskog, 1983) as
\begin{equation}
\begin{split}
J &= |\Im(V_{ik}V^*_{jk}V_{il}V^*_{jl})| \qquad \text{no sum, } i\neq j \neq k \neq l \\
&=c_1s_1c_2^2s_2c_3s_3\sin\delta \\
&\approx A^2\lambda^6\eta \qquad  \qquad  \qquad \text{in the Wolfenstein parametrisation}.
\end{split}
\end{equation}
For $J=0$, there is no CP violation. Even though $\delta$ is large, CP violation is very small, as $\lambda^6 \approx 0.0001$. It is very difficult to get CP violation. You need:
\begin{itemize}
\item $\delta \neq 0,\ \frac{\pi}{2}$ complex phase;
\item $\theta_1$, $\theta_2$, $\theta_3$ $\neq 0,\ \frac{\pi}{2}$, no trivial angles (otherwise you can rotate the phase away);
\item all masses $\neq 0$, otherwise $N=3$ reduces to $N=2$ and you can rotate the phase away;
\item the masses must be different between up-type and down-type quarks, otherwise $N=3$ reduces to $N=2$ and you can rotate the phase away;
\item at least two interfering amplitudes (because physics $\sim |\mathcal{M}|^2)$;
\item a flavour changing loop, involving all three generations.
\end{itemize}
%
\subsection{$K-\bar{K}$ Mixing and Decay}
%
Remarkably, CP violation was observed already in 1964 (Cronin \& Fitch) through study of decays of neutral $K$ mesons, $K^0=(d\bar{s})$ and $\bar{K}^0=(s\bar{d})$. (CP violation has also been observed in B physics in 2001.) Under CP, $K^0 \leftrightarrow \bar{K}^0$, conventially written as:
\begin{equation}
CP|K^0\rangle = -|\bar{K}^0 \rangle, \qquad CP|\bar{K}^0 \rangle = -|K^0\rangle.
\end{equation}
The CP eigenstates are
\begin{equation}
|K_1\rangle = \frac{1}{\sqrt{2}}(|K^0\rangle - |\bar{K}^0\rangle), \qquad |K_2\rangle = \frac{1}{\sqrt{2}}(|K^0\rangle + |\bar{K}^0\rangle),
\end{equation}
with eigenvalues $+1$ and $-1$ respectively. Kaons decay to combinations of pions:
\begin{equation}
\begin{split}
|\pi^0\pi^0\rangle\ \text{and} \ |\pi^+\pi^-\rangle \qquad &\text{with }CP=+1, \\
|\pi^0\pi^0\pi^0\rangle\ \text{and} \ |\pi^+\pi^-\pi^0\rangle \qquad &\text{with }CP=-1,
\end{split}
\end{equation}
because the pion has odd parity. So if there is no CP violation, $K_1 \to \pi\pi$ and $K_2 \to \pi\pi\pi$. The two pion state has a large phase space so the decay is quick. Converseley, the three pion state has a smaller phase space and the decay is slow ($m_K \approx 495$ MeV whilst $m_\pi \approx 140$ MeV. In reality, there are two \textit{physical} states, which we can observe: $K_S$ ("$K$-short") and $K_L$ ("$K$-long"). Usually, these correspond to 
\begin{equation}
\begin{split}
K_S \to \pi\pi \qquad \textit{i.e. } &K_S \approx K_1, \\
K_L \to \pi\pi\pi \qquad \textit{i.e. } &K_L \approx K_2.
\end{split}
\end{equation}
However, sometimes $K_L \to 2\pi$ and $K_S \to 3\pi$:
\begin{equation}
\eta_{+-} = \frac{\langle\pi^+\pi^-|\mathcal{H}|K_L\rangle}{\langle \pi^+\pi^- |\mathcal{H}|K_S \rangle}, \qquad 
\eta_{00} = \frac{\langle\pi^0\pi^0|\mathcal{H}|K_L\rangle}{\langle \pi^0\pi^0 |\mathcal{H}|K_S \rangle},
\end{equation}
and $\eta_{00} \sim \eta_{+-} \sim 10^{-3}$. Clearly this violates CP. There are two possibilities to explain the observation:
\begin{enumerate}
\item \textbf{Direct CP Violation: }

$K_S,\ K_L$ are CP eigenstates, but the decay process is CP violating.
\begin{figure}[H]
  \subfloat[$\Delta S =1,\ CP$]{\includegraphics[width=0.5\textwidth]{figs/49a.png}}
  \hfill
  \subfloat[$\Delta S =1, \ CP$ violating if $J\neq 0$]{\includegraphics[width=0.5\textwidth]{figs/49b.png}}
  \end{figure}
\item \textbf{Indirect CP Violation: }

$K_S$ and $K_L$ are mixtures of $K_1$ and $K_2$, through interactions given by box diagrams. 
\begin{figure}[H]
  \subfloat[$\Delta S =2,\ CP$ violating if  $J\neq 0$]{\includegraphics[width=0.5\textwidth]{figs/49c.png}}
  \hfill
  \subfloat[$\Delta S =2,\ CP$ violating if  $J\neq 0$]{\includegraphics[width=0.5\textwidth]{figs/49d.png}}
  \end{figure}
\end{enumerate}
Let's focus on indirect CP violation, via kaon mixing. From the CPT theorem,
\begin{equation}
\begin{split}
&\langle K^0 | \mathcal{H} | K^0 \rangle = \langle \bar{K^0} | \mathcal{H} | \bar{K^0} \rangle \\
&\langle K^0 | \mathcal{H} | \bar{K^0} \rangle = \langle \bar{K^0} | \mathcal{H} | K^0 \rangle,
\end{split}
\end{equation}
where the first line is determined by QCD and the second line is determined by electroweak for $\Delta S = 2$. We can write the mass matrix
\begin{equation}
\begin{pmatrix}
M & M_X \\
M_X^* & M 
\end{pmatrix},
\qquad |M_X| \ll M \equiv m_K,
\end{equation}
where $m_K \sim 495$ MeV is the mass of the kaon. The matrix is hermitian so has real eienvalues $M \pm |M_X|$, with eigenvectors
\begin{equation}
\frac{1}{\sqrt{2}} (e^{i\frac{\phi}{2}}|K^0\rangle \pm e^{-i\frac{\phi}{2}}|\bar{K^0}\rangle),
\end{equation}
where $M_X = |M_X|e^{i\phi}$. 

So if there is no CP violation, then $\phi=0$, which means: $|K_L\rangle = |K_2\rangle$ with mass $M+|M_X|$, and $|K_S\rangle = |K_1\rangle$ with mass $M-|M_X|$. We can write $\Delta m_K = m_{K_L} - m_{K_S} = 2|M_X| = 2 \Re M_X$.

If there is CP violation, then $\phi\neq 0$, but we assume it is small. Then we can write
\begin{equation}
\begin{split}
|K_L\rangle &= |K_2\rangle + \frac{i\phi}{2}|K_1\rangle \\
|K_S\rangle &= |K_1\rangle + \frac{i\phi}{2}|K_2\rangle,
\end{split}
\end{equation}
so $|K_L\rangle$ has an admixture of $|K_1\rangle$, and can therefore decay to $\pi\pi$. We can define the parameter $\epsilon$ as a measure of indirect CP violation:
\begin{equation}
\epsilon = \frac{i\phi}{2} = \frac{i\Im M_X}{2\Re M_X} = \frac{i\Im M_X}{\Delta m_K}.
\end{equation}
\subsubsection{Notes:}
\begin{itemize}
\item This analysis is \textit{very} oversimplified! In practice we need to account also for the width $\Gamma$, so $M\to M + \frac{i\Gamma}{2}$, making things rather more complicated. In this case we find
\begin{equation}
\epsilon \approx e^{i\theta_\epsilon}\sin\theta_\epsilon \frac{\Im M_X}{\Delta m_K}, \qquad \tan\theta_\epsilon = \frac{2\Delta m_K}{\Gamma_s - \Gamma_L}, \qquad \theta_\epsilon \approx 45 \degree.
\end{equation}
\item We can compute $M_X$, and thus $\Delta m_K$ and $\epsilon$, from the box diagram:
\begin{equation}
M_X = \frac{\langle K^0 | \mathcal{H} | \bar{K}^0 \rangle}{2m_K},
\end{equation}
where the denominator is a normalisation factor due to density of state arguments (see the QFT course from last term). Calculating the phase of $\epsilon$, and the equivalent parameter $\epsilon^\prime$ from \textit{direct} CP violation is rather harder.
\item In B physics, direct CP violation is more important.
\end{itemize}
%
\subsection{Box Diagrams}
%
Let's revisit the box diagrams we saw before.
\begin{figure}[H]
  \subfloat[]{\includegraphics[width=0.5\textwidth]{figs/49c.png}}
  \hfill
  \subfloat[]{\includegraphics[width=0.5\textwidth]{figs/49d.png}}
  \end{figure}
We consider the incoming momenta $p=0$ (because we want to calculate the mass). We will work in the Feynman-'t Hooft gauge, so we also need to consider the Goldstone boson contributions: there are three for each diagram, for example the additions to the first diagram are depicted below.
\newline
  \includegraphics[width=0.4\linewidth]{figs/51a.png}
\newline
Let's concentrate on the first diagram, with two $W$s. The matrix element can be written
\begin{equation}
\begin{split}
\mathcal{M}_{WW} = &\sum_{(a,b)=u,c,t} \bigg(\frac{-ig}{\sqrt{2}}\bigg)^4 V_{as}V^*_{ad} V^*_{bd}V_{bs} \int \frac{d^4l}{(2\pi)^4}\frac{i^4}{(l^2-m_W^2)^2(l^2-m_a^2)(l^2-m_b^2)} \\
&\times \bar{u}_d \gamma^\mu\frac{1}{2}(1-\gamma^5)(\slashed{l}+m_a)\gamma^\nu \frac{1}{2}(1-\gamma^5) u_s \bar{v}_d \gamma_\nu \frac{1}{2}(1-\gamma^5)(-\slashed{l}+m_b)\gamma_\mu \frac{1}{2}(1-\gamma^5)v_s
\end{split}
\end{equation}
